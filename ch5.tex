\chapter{Conclusions}
\TODO{
We have introduced a framework for classifying ion adsorption at lipid bilayers based on the degree of dehydration observed in molecular dynamics simulations. This classification distinguishes between steric, imperfect, and perfect adsorption modes, corresponding to fully hydrated, partially dehydrated, and completely dehydrated ions within the hydration boundary of the bilayer.

Our results show that the electric field at the hydration shell of an ion correlates strongly with its observed mode of adsorption. Ions with high field strength, such as \mg, remain hydrated and adsorb sterically. In contrast, \na\ and \li\ exhibit lower field strengths and bind with partial or full dehydration.

These adsorption modes are predictive of changes to bilayer structure. Systems with larger populations of imperfect or perfectly adsorbed ions display increased lipid chain order and hydrocarbon thickness. Systems dominated by steric adsorption show bilayer structures closer to the no-salt case.

The observed trends hold across multiple force-field parameterizations, including those with differing water-exchange kinetics for \mg. Poisson--Boltzmann modeling confirms that the accumulation of ions within the hydration boundary is distinct from diffuse ionic layering predicted in the bulk.

Altogether, this work provides a mechanistic link between ion properties and lipid structural response, grounded in simulation observables. It suggests that the strength of ion adsorption---and its structural consequences---can be anticipated from the ion's electric field without requiring direct tuning to experimental constraints.
}
\bibliography{refs}
\end{document}


