\documentclass{article}
\usepackage[letterpaper,margin=1in,footskip=0.25in]{geometry}
\fontfamily{cmr}
\usepackage[final]{changes}
\usepackage{lipsum}
\usepackage[nottoc]{tocbibind}
\usepackage[T1]{fontenc}
\usepackage[utf8]{inputenc}
\usepackage{flafter}
\usepackage{floatrow}
\usepackage{float}
\floatstyle{plaintop}
\restylefloat{table}
%\usepackage{floatflt}

\usepackage{placeins}
\usepackage{siunitx}
\usepackage{graphicx}
\graphicspath{{figures/}}% Include figure files
\usepackage{dcolumn}% Align table columns on decimal point
\usepackage{appendix}
\usepackage{amsmath}  % usually already loaded
\newcommand{\norm}[1]{\left\lVert#1\right\rVert}
\usepackage{cases}
\usepackage{calc}
\usepackage{amssymb}
\usepackage{color}
\usepackage{rotating}
\usepackage{enumitem}
\usepackage{soul}
\usepackage{indentfirst}
\usepackage{hyphenat}
\usepackage{xspace}
\usepackage{subcaption}
\usepackage{booktabs}
\usepackage{multirow}
\usepackage{tabularx}
\newcolumntype{Y}{>{\centering\arraybackslash}X}
\usepackage{xcolor}
\usepackage{lineno}
\usepackage{setspace}
\usepackage{titlesec}
\usepackage{hyperref}
\usepackage[superscript]{cite}
\usepackage{lineno}
\newcommand{\sigmaij}{$\sigma_{ij}$}
\newcommand{\epsilonij}{$\epsilon_{ij}$}
\newcommand{\db}{$\text{D}_\text{B}$}
\newcommand{\dhh}{$\text{D}_\text{hh}$}
\newcommand{\dc}{$\text{2D}_\text{C}$}
\newcommand{\al}{$\text{A}_{\text{L}}$}
\newcommand{\vl}{$\text{V}_{\text{L}}$}
\newcommand{\vh}{$\text{V}_{\text{H}}$}
\newcommand{\vc}{$\text{V}_{\text{C}}$}
\newcommand{\sig}{$\sigma$}
\newcommand{\eps}{$\epsilon$}
\newcommand{\na}{Na\textsuperscript{+}}
\newcommand{\li}{Li\textsuperscript{+}}
\newcommand{\mg}{Mg\textsuperscript{2+}}
\newcommand{\cl}{Cl\textsuperscript{-}}
\newcommand{\mgcl}{MgCl\textsubscript{2}}
\newcommand{\nacl}{NaCl}
\newcommand{\licl}{LiCl}
\newcommand{\mbnbfix}{MB-NB-fix~}
\newcommand{\mbnbfixmicro}{Grotz \etal~} %CHANGED THIS
\newcommand{\nambnbfix}{Na\textsuperscript{+}--Saunders \etal}
\newcommand{\limbnbfix}{Li\textsuperscript{+}--Joung and Chetatham}
\newcommand{\mgmbnbfix}{Mg\textsuperscript{2+}--Li \etal} %CHANGED THIS
\newcommand{\mgmicro}{Mg\textsuperscript{2+}--Grotz \etal}
\newcommand{\mglbrules}{Mg\textsuperscript{2+}--Li \etal}

\begin{document}
Accurately modeling ion--lipid interactions remains a persistent challenge in computational biophysics.
Alkali and alkaline earth metal cations associate with zwitterionic membranes, modulating the electrostatic
potential near the bilayer and influencing lipid packing, water structure, and the behavior of membrane-
associated proteins. These subtle but consequential effects impact processes such as signaling, fusion, and
membrane phase behavior. However, widely used molecular dynamics (MD) model parameters (force-fields) often differ substantially
in their predictions of ion dehydration and coordination.

My dissertation focuses on improving and evaluating ion--lipid interaction terms in force-fields.
I have developed parameters for \na{} and \li{}, and am currently characterizing two distinct \mg{} models
in simulations of POPC and DMPC bilayers. Each \mg{} parameter set (2024 and 2025) is paired with
two water--ion interaction models developed by other groups: one reproducing hydration free energies and the
other tuned to match experimental water residence times. 

The cation--lipid interaction terms were developed from DFT-optimized clusters using ligands representing the 
ester and phosphate fragments of the lipid.
In our newer (2025) parameterization of \mg, we preserve sixfold coordination by replacing missing ligands with
water--an approach adapted from recent work with \mg{} and ATP. Unexpectedly, simulations of DMPC
with the 2025 model resulted in a larger number of \mg{} adsorbed, despite DMPC having a smaller area per lipid than POPC.
Tighter packing was expected to increase inner-shell coordination between \mg{} and lipid oxygens, but not necessarily increase adsorption overall. 
This result has prompted a comparative study using the 2024 model in the same DMPC
systems, which is currently ongoing and represents the final analysis needed to complete my dissertation.

Beyond resolving force field behavior, this project supports a longer-term goal: enabling multiscale
simulations of membrane organization. Our lab has previously used this lipid force field to develop parameters for mean-field modeling of
cholesterol--phospholipid systems and systems containing lipid-like molecules such as sphingomyelin.
However, we lacked reliable and experimentally validated parameters for ion--lipid interactions, 
preventing their inclusion in these models. My current work addresses this limitation by
identifying which ion observables--adsorption modes, and adsorbed charge density--may
translate into new terms in mesoscopic models. Together with our validated monovalent ion models, these
results will lay essential groundwork for future study of membrane, and membrane protein dynamics and organization 
at the mesoscopic and cellular scale.

During the Dissertation Completion Fellowship semester, I will complete the analysis of the DMPC+2024 simulations, 
finalize the POPC/DMPC comparison, and integrate this work into my dissertation’s
final chapter. I will also finish and submit the associated manuscript, prepare for my defense, and complete
formatting and final revisions. I expect to defend by late September 2025 and submit my final ETD by early
October.

%Accurate modeling of ion--lipid interactions remains a persistent challenge in computational biophysics.
%Alkali and alkaline earth metal cations associate with zwitterionic membranes, 
%modulating electrostatic potential near the bilayer and influencing lipid packing, water structure, and the behavior of membrane-associated proteins.
%These subtle but consequential effects propagate into processes such as signaling, fusion, and membrane phase behavior.
%Yet, computational model parameters (force-fields) vary widely in their prediction of ion dehydration and coordination.
%
%My dissertation focuses on development and evaluation of ion--lipid interaction terms in molecular dynamics models.
%I have developed interaction models for \na~and \li, and am currently characterizing two Mg\textsuperscript{2+} 
%parameter sets in simulations of 1-palmitoyl-2-oleoyl-sn-glycero-3-phosphocholine (POPC) and 
%1,2-dimyristoyl-sn-glycero-3-phosphocholine (DMPC) bilayers,
%using two distinct water--ion interaction models developed by other groups:
%one designed to match hydration free energies, and another tuned to reproduce experimentally motivated water residence times in the Mg\textsuperscript{2+} first coordination shell.
%The Mg\textsuperscript{2+} parameters themselves were developed using ion clusters optimized with density functional theory (DFT),
%where model lipid ligands were arranged to reflect typical phosphate and ester coordination environments.
%In our more recent (2025) parameterization, the clusters retain sixfold coordination 
%in systems with fewer ligands by substituting water for missing sites--an approach adapted from recent Mg\textsuperscript{2+}--ATP parameterization strategies.
%This refinement led to unexpected behavior:
%simulations using the 2025 Mg\textsuperscript{2+} model produced significantly more adsorption in DMPC than anticipated,
%despite its smaller surface area-per-lipid.
%While greater inner-shell coordination between ions and lipid oxygens was expected in DMPC, the total number of adsorbed ions was higher than expected.
%These results have motivated a comparison with our previously published 2024 Mg\textsuperscript{2+} model,
%which is currently being tested using both water--ion interaction models.
%This work is already underway and represents the final stage of analysis needed to complete my dissertation.
%
%This project also supports a longer-term goal of enabling multiscale membrane modeling.
%Our group previously applied this lipid force-field to cholesterol--phospholipid systems, and systems with lipid-like molecules like sphingomyelin, aiming to construct mesoscopic models of lateral domain formation.
%However, the lack of reliable and experimentally-validated atomistic data on ion--lipid interactions prevented us from incorporating cations into our coarse-grained model.
%The current work addresses that gap by identifying which observables--such as adsorption modalities, the quantity and density of adsorbed charges, and acyl-chain ordering—show patterns
%that could inform new interaction terms in mesoscopic models.
%Together with insights from our monovalent ion parameterizations, these data will lay 
%essential groundwork for future efforts to simulate lateral membrane organization and protein distribution at whole-cell scales.
%
%During the Dissertation Completion Fellowship semester,
%I will complete the analysis of the DMPC + 2024 Mg\textsuperscript{2+} simulations,
%finalize the comparative POPC/DMPC ion binding study,
%and incorporate this work into my final dissertation chapter.
%I will also complete and submit the accompanying manuscript,
%prepare for my defense,
%and finalize formatting and revisions.
%With this timeline,
%I expect to defend by late September 2025 and submit my final ETD by early October.


%This project addresses the need for accurate molecular modeling of ion interactions with lipid membranes
%— a persistent challenge in biophysical chemistry. Divalent cations like Mg\textsuperscript{2+} can associate
%with zwitterionic lipids, modulating the electrostatic potential near the membrane and influencing water
%structure, lipid packing, and interactions with membrane-associated proteins. These subtle but significant
%changes have downstream effects on processes such as signaling, fusion, and membrane phase behavior.
%However, force fields vary widely in how they model these interactions, particularly with respect to ion
%dehydration and coordination, making reliable simulation challenging.
%
%My dissertation focuses on evaluating and comparing two Mg\textsuperscript{2+} parameterizations across
%POPC and DMPC bilayers, using two different water--ion interaction models: one designed to match hydration
%free energies, and another optimized to reproduce realistic water residence times in the ion’s first coordination
%shell. POPC and DMPC share the same headgroup but differ in acyl-chain composition and packing, offering a
%useful contrast in membrane structure while controlling for headgroup chemistry.
%
%Our group has previously used this force field family to study cholesterol--POPC mixtures, with the goal of
%constructing mesoscopic models for lateral domain formation. However, the lack of reliable atomistic data on
%ion--lipid interactions prevented us from incorporating divalent cation effects into coarse-grained models.
%This project addresses that gap by producing well-characterized atomistic simulations of Mg\textsuperscript{2+}
%binding to simple membranes, identifying which observables (e.g., ion density profiles, adsorption depth,
%and chain order) are robust across conditions and which are sensitive to force field choice. Together with
%insights from our monovalent ion parameterization efforts, these results can serve as inputs or constraints
%for future coarse-grained or continuum models --- a step toward simulating membrane behavior at whole-cell
%scales.
%
%Recent results from DMPC simulations using the 2025 Mg\textsuperscript{2+} model showed a surprisingly high
%degree of ion adsorption, despite the smaller area per lipid in DMPC that would normally reduce surface
%availability. While increased inner-shell coordination between ions and lipids was expected, the total number
%of ions bound to the bilayer was much higher than anticipated. This observation triggered a deeper comparison
%with our earlier Mg\textsuperscript{2+} parameterization, which is now underway using 1000\,ns simulations
%with both water--ion interaction models. The results of this comparison will complete my final dissertation
%chapter.
%
%\vspace{1em}
%\noindent\textbf{Timeline:}
%\begin{itemize}
%  \item \textbf{Summer 2025 (pre-fellowship):} Complete 1000\,ns DMPC + 2024 Mg\textsuperscript{2+}
%    simulations; analyze coordination behavior and compare to POPC; finalize and submit POPC
%    Mg\textsuperscript{2+} paper characterizing parameterization choices; draft and revise the
%    DMPC/POPC dissertation chapter.
%  \item \textbf{Fall 2025 (fellowship semester):} Complete the DMPC/POPC paper; integrate chapter
%    into dissertation; revise with advisor feedback and finalize formatting; submit to committee and
%    prepare defense; defend in late September 2025; submit final ETD by early October.
%\end{itemize}


%Studying cellular membranes is a complex task, as they have many components and are exposed to very complex and inhomogeneous solutions. If we aim to understand
%these structures, we must understand all of the component parts individually so that we can eventually study how they interact. One major component of these
%membranes are the phospholipids themselves -- these are not a passive member of the structure, but are responsible for structure and two-dimensional
%organisation (such as lipid-rafts which can sequester specific proteins for functional purposes). Phase behavior and raft forming has been studied extensively in pure solvent,
%and effects from different lipids have been studied as well. However, the effect of ions on the formation of these rafts has not been studied substantially. Simulation studies of this
%kind have been limited by our models of ion-lipid interactions. To this end, we have developed a method to produce lennard-jones interaction cross-terms to produce parameters that
%improve the local-interactions in clusters of ions and small molecule lipid mimetics, which can have significant effects on the larger condensed phase simulations. In simulations with \na and POPC
%we not improved reproduction of experimental SAXS and SANS results. However, when we turn to a divalent ion \mg, we find that experimental results at the concentrations of ions that are required
%for sampling in simulations are limited. Thus, we have tried several methods to develop interaction cross terms -- following our methodology with \na and \li, and also following the method used
%to develop ion-DNA interaction terms by our group. These result in parameters that give very different behavior in lipids. We note that \na and \li tend to adsorb in a way that
%dehydrates the ion and replaces those waters with lipid parts, which we call Langmuir-type adsorption. \mg tends to adsorb in a much more steric modality, remaining in the lipid bilayer
%region but not dehydrating substantially. However, our more recently developed parameters combined with a \mg-water interaction term developed to increase water exchange from the first coordination shell
%of \mg result in a substantial number of ions losing one/two water molecules within the bilayer surfacial region. We note a trend -- that the number of charges adsorbed in non-steric modes tend to
%perturb the bilayer structure such that the bilayer is thicker than the equivalent no-salt simulations. 
%We currently are working on understanding the differences between these parameters, with the hope to improve our understanding of how small-molecule interactions lead to results in the lipid bilayer.
%Additionally, to further understand these parameters we have started to simulate a different lipid, DMPC, which packs more closely than POPC in pure water. This will inform us on how the pertubation 
%from these adsorbed charges depends on the type of lipid in the bilayer. This all together is a significant step towards understanding how these simulations behave

\end{document}
